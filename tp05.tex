%%% LaTeX Template: Article/Thesis/etc. with colored headings and special fonts
%%%
%%% Source: http://www.howtotex.com/

\documentclass[12pt]{article}


\usepackage{apuntes-estilo}
\usepackage{fancyhdr,lastpage}
\usepackage{verbatim}



\def\maketitle{

% Titulo 
 \makeatletter
 {\color{bl} \centering \huge \sc \textbf{
Trabajo práctico N 5 \\
\large \vspace*{-8pt} \color{black} Inicio y apagado, repaso. 
 \vspace*{8pt} }\par}
 \makeatother


% Autor
 \makeatletter
 {\centering \small 
	Introducción a la Administración de Sistemas \\
 	Departamento de Ingeniería de Computadoras \\
 	Facultad de Informática - Universidad Nacional del Comahue \\
 	\vspace{20pt} }
 \makeatother

}

% Custom headers and footers
\fancyhf{} % clear all header and footer fields
\fancypagestyle{plain}{\fancyhf{}}
  	\pagestyle{fancy}
 	\lhead{\footnotesize TP N 5 - Inicio y apagado, repaso. }
 	\rhead{\footnotesize \thepage\ }	% "Page 1 of 2"

\def\ti#1#2{\texttt{#1} & #2 \\ }



\begin{document}

\thispagestyle{empty}
\maketitle
\setlength{\parindent}{0pt}

\paragraph{Los siguientes ejercicios se realizan en su totalidad en el equipo asignado al grupo.
Utilice la redirección y el editor \texttt{vi} para guardar las salidas y editar los resultados 
en un archivo llamado trabajo-practico-5.txt . En el archivo de resolución indique su nombre y apellido, 
y el número de ejercicio a resolver, junto a la salida incluyendo el comando que ejecuta. }

\section*{Ejercicio 1}

\begin{enumerate}
\item El gestor de arranque utilizado por las máquinas del laboratorio se llama GRUB. Investigue 
dónde se encuentran los archivos de configuración del mismo dentro del sistema. 
\item Explique con sus palabras cuál es la función de dicho software.
\end{enumerate}

\section*{Ejercicio 2}
\begin{enumerate}
\item El archivo \texttt{/etc/inittab} contiene la configuración del proceso 
\texttt{init}. Utilizando el manual de dicho archivo, determine cuál es el 
nivel de ejecución predeterminado del sistema. Copie en su resolución la línea 
correspondiente a dicha configuración. 
\item Obtenga el nivel de ejecución actual. ¿Se encuentra el sistema en el nivel 
de ejecución predeterminado?. ¿Qué comando utilizó? 
\item Utilizando el comando \texttt{telinit} cambie el nivel de ejecución al de usuario 
individual. 
	\begin{itemize}
	\item ¿Cón qué argumentos ejecutó el comando telinit?
	\item ¿Qué sucedió y por qué cree que sucedió?
	\item ¿Se encuentra la máquina apagada? (intente un ping). 
	\item ¿Qué pasos deben llevarse a cabo para volver al estado de multiusuario? (solicite asistencia a la cátedra para resolver este punto).
	\end{itemize}
\item Utilizando el comando \texttt{shutdown} cambie a nivel de ejecución seis, pero 
con un tiempo de postergación de diez segundos, y tal que se envíe el mensaje 
``Estaremos de vuelta en unos instantes'' en todas las terminales del sistema. ¿Qué comando ejecutó? ¿Qué sucedió?
\item Vea el siguiente video : http://youtu.be/JzzsFVMEU6I  ; y trate de identificar : 
\begin{itemize}
\item ¿En qué momento está en ejecución el gestor de arranque? ¿Cual es el gestor de arranque?
\item ¿En qué momento comienza la carga del kernel Linux? 
\item ¿En qué momento comienza la ejecución del proceso \textbf{init}?
\item Observe algunos de los servicios iniciados por init.
\end{itemize}
\end{enumerate}

\section*{Ejercicio 3 - Sistema de Archivos}
Sobre los directorios \texttt{/bin/} y \texttt{/usr/bin/}

\begin{itemize}
\item - ¿Qué tipo de archivos hay en esos directorios?
\item - ¿Qué diferencia existe entre los dos directorios?
\item - ¿Cuáles son los permisos del directorio \texttt{/usr/bin}?
\item - ¿Quien puede modificar el contenido de ese directorio?
\item - ¿Por qué el directorio debe tener esos permisos, y no otros?
\end{itemize}

\item - ¿Qué tipo de archivos hay en el directorio \texttt{/etc/}?

\item - ¿Qué tipo de archivos hay en el directorio \texttt{/tmp/}?

\item - ¿Qué tipo de archivos hay en el directorio \texttt{/var/}?

\section*{Ejercicio 4 - Procesos}
Ayuda: \texttt{ps ; man ps}
\begin{enumerate}
\item Obtenga la lista de \textit{todos} los programas en ejecución, incluyendo al menos la siguiente información:
dueño del proceso, identificador de proceso (PID), identificador del padre (PPID),
comando que inició el proceso. Guarde esta salida en un archivo llamado \texttt{tp02.txt}.
\item De la lista anterior, ¿Cuál es el primer proceso del sistema? ¿Cual es su PID?
\item ¿Cuántos procesos hay en ejecucion? Ayuda : \texttt{wc}
\end{enumerate}

\section*{Ejercicio 5 - Entorno del shell}
Ayuda: \texttt{env}
\begin{enumerate}
\item Obtenga la lista de variables de entorno del usuario.
\item ¿Cuántas variables hay definidas? Ayuda : \texttt{wc}
\item Obtenga únicamente el valor de las variables HOME. ¿Qué comando utilizó para obtener la respuesta?
\item Obtenga únicamente el valor de las variables PATH. ¿Qué comando utilizó para obtener la respuesta?
\end{enumerate}


\section*{Ejercicio 6 - Sesión de trabajo}

\item ¿Qué comando utiliza para saber en qué directorio se encuentra trabajando?
\item ¿Cómo saber cuáles son las opciones soportadas por un comando sin recurrir a Internet?

\item Realice las siguientes acciones (especifique en el archivo de resolucion del trabajo práctico todos los comandos que utiliza) : 
	\begin{enumerate}
	\item Listar el contenido del directorio actual con los detalles de cada archivo. 
	\item Crear un directorio llamado \texttt{backups} en su directorio HOME.
	\item Crear un directorio llamado \texttt{vacio} en su directorio HOME.
	\item Copiar todos los archivos de su directorio al directorio creado \texttt{backups/}.
	\item Copiar el archivo \texttt{/etc/passwd} al directorio \texttt{vacio/}.
	\item Eliminar un archivo en su directorio HOME.
	\item Eliminar el directorio \texttt{vacio/}
	\item Observar el contenido del archivo \texttt{/etc/services}, paginando la salida en la terminal (mencione al menos dos maneras de hacerlo). ¿Qué dice la última línea de ese archivo?
	\end{enumerate}
\end{enumerate}



\section*{Ejercicio 4}
El comando \texttt{du} se puede utilizar para ver el espacio utilizado por archivos y directorios.
Si ejecuta \texttt{du -h /etc} el sistema muestra de manera legible (el argumento \texttt{-h} significa "legible para las personas")
el espacio utilizado en el directorio \texttt{/etc} (suma los tamaños de todos los archivos y directorios que estén dentro de \texttt{/etc}).

\begin{itemize}
\item - ¿Cuanto espacio utiliza el directorio \texttt{/usr}?
\item - ¿Porqué el directorio \texttt{/usr} ocupa mas espacio que \texttt{/etc}?
\item - ¿Cuanto espacio utiliza el directorio raíz \texttt{/} ?
\end{itemize}


\section*{Ejercicio 5}
\begin{enumerate}
\item El comando \texttt{hostname} se utiliza para conocer el nombre del sistema (o para definirlo).
\begin{itemize}
\item - ¿Qué nombre tiene su sistema?
\end{itemize}
\end{enumerate}

\begin{enumerate}
\item El programa \texttt{hostname} viene empaquetado en un paquete de nombre homónimo.
El comando \texttt{dpkg -L hostname}  lista los archivos que el paquete \texttt{hostname} instaló en el sistema.
\begin{itemize}
\item - Indique que son cada uno de los archivos listados (por ejemplo: si son binarios ejecutables, archivos de configuración, archivos de log, etc).
\end{itemize}
\end{enumerate}



\section*{Ejercicio 6}
El archivo \texttt{/proc/meminfo} contiene información de la memoria utilizada en kilobytes.

\begin{itemize}
\item - Utilice cat para ver la información.
\item - ¿Cuanta memoria total en MegaBytes tiene el sistema?
\item - ¿Cuanta memoria se está utilizando del total? (en MegaBytes). Ayuda: utilice el dato MemFree
\end{itemize}


\section*{Ejercicio 7}
En el directorio \texttt{/proc} existen directorios con nombres "numéricos", que corresponden con el PID de los procesos del sistema.
Estos directorios contienen información de los procesos (de los programas en ejecución).
\begin{itemize}
\item - Liste los directorios en \texttt{/proc}, en el cual, su nombre, comience con un número (liste solo los nombres de los directorios, no su contenido). Ayuda : opción \texttt{-d} de \texttt{ls}.
\item - Cuente con \texttt{wc -l} el listado anterior. ¿Cuántos procesos tiene en ejecución el sistema?.
\item - Ejecute \texttt{ps -ef} , y \texttt{ps -ef | wc -l} . ¿Qué reporta \texttt{ps -ef}?
\item - ¿Coincide la cantidad de procesos reportados con \texttt{ps -ef} con el obtenido en el primer inciso de este ejercicio?
\end{itemize}


\section*{Ejercicio 8}
El comando \texttt{find /var} lista todos los archivos y directorios que existen debajo de \texttt{/var}.
\begin{itemize}
\item - ¿Cuantos archivos y directorios contiene el \texttt{/var} de su sistema?. Ayuda : \texttt{wc}.
\item - ¿Cuanto espacio utiliza en disco todo el \texttt{/var}?. Ayuda: \texttt{du}.
\item - ¿Cuál es el directorio en donde el sistema guarda los mensajes de log?
\item - Mencione 5 archivos de log de su sistema.
\end{itemize}

\end{document}
