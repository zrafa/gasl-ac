%%% LaTeX Template: Article/Thesis/etc. with colored headings and special fonts
%%%
%%% Source: http://www.howtotex.com/

\documentclass[12pt]{article}


\usepackage{apuntes-estilo}
\usepackage{fancyhdr,lastpage}



\def\maketitle{

% Titulo 
 \makeatletter
 {\color{bl} \centering \huge \sc \textbf{
Administrando Cuentas de Usuario \\
% \large \vspace*{-8pt} \color{black} Vim el Editor de Seis Billones de Dólares
 \vspace*{8pt} }\par}
 \makeatother


% Autor
 \makeatletter
 {\centering \small 
 	Departamento de Ingeniería de Computadoras \\
 	Facultad de Informática - Universidad Nacional del Comahue \\
 	\vspace{20pt} }
 \makeatother

}

% Custom headers and footers
\fancyhf{} % clear all header and footer fields
\fancypagestyle{plain}{\fancyhf{}}
  	\pagestyle{fancy}
 	\lhead{\footnotesize  Administrando Cuentas de Usuario Linux - Departamento de Ingeniería de Computadoras}
 	\rhead{\footnotesize \thepage\ }	% "Page 1 of 2"

\def\ti#1#2{\texttt{#1} & #2 \\ }



\begin{document}

\thispagestyle{empty}
\maketitle
\setlength{\parindent}{0pt}




%\textit{Similitudes entre administradores de sistema y
%narcotraficantes: ambos miden cosas en Kilos y tienen usuarios} (Viejo y
%cansador chiste de computación)
\section{Introducción}
La creación y borrado de usuarios es una tarea de rutina bastante 
común dentro de la administración de sistemas. Por este motivo, 
es necesario familiarizarse con varios aspectos relativos a las
cuentas de los usuarios, por lo que en este apunte se detallan 
las tareas de creación, modificación y borrado de usuarios dentro 
de un sistema GNU/Linux.

Del mismo modo las cuentas de usuario y su mantenimiento, son un 
factor clave en la administración de la seguridad del sistema.

Por otra parte, debido a la diversidad de distribuciones existentes,
las herramientas disponibles para la gestión de cuentas de usuario pueden 
variar de unas a otras (consulte la documentación
específica a su distribución GNU/Linux para obtener mayor información). Sin 
embargo, los cambios que estas herramientas implementan  en última instancia, 
son, en general, los mismos en todas las distribuciones.  

\subsection{ ¿Qué es una cuenta?}
Concretamente es un nombre de usuario mas una contraseña, salvo
excepciones, y todos los archivos (de configuración y aquellos personales) que
impliquen el ingreso y permanencia de un usuario en el sistema.

Cuando una computadora la usa mucha gente es usualmente necesario hacer
diferencias en estos usuarios. Por ejemplo, para que sus archivos privados
permanezcan privados. Esto es importante aun si el sistema es usado por una
sola persona a la vez\footnote{Recuerde que los sistemas GNU/Linux son 
en esencia de tiempo compartido, esto es, múltiples usuarios pueden ejecutar
procesos concurrentemente en un dado momento, sin interferir unos con otros.},
como sucede con la mayoría de las computadoras personales.
\footnote{Podría ser un poco embarazoso si mi madre o mi hermana pudiera leer 
mis cartas de amor.} 

Es así que, a cada usuario se le da un nombre de usuario 
\textbf{único}, y ese nombre es
usado para ingresar al sistema. Además del nombre, el sistema le asigna 
un identificador numérico único o \texttt{uid} (del inglés user identification). 
Luego una \textit{cuenta} es todos los archivos, recursos, e información 
que pertenece a un usuario. El término insinúa algo similar a 
a lo que ocurre en bancos y en sistemas comerciales, en donde cada cuenta 
usualmente tiene algo de dinero asignado, y ese dinero se gasta a 
diferentes velocidades dependiendo de cuantos usuario exijan el sistema.
Dentro de un sistema operativo, una cuenta, por ejemplo  puede tener una 
cantidad de espacio de disco que puede tener un precio por mega por día, y
tiempo de procesamiento que puede tener un precio por segundo, por citar algunos
ejemplos. 

\section{ Crear una cuenta de usuario}
El núcleo de Linux en sí trata a los usuario como meros números. Cada
usuario es identificado por un único numero entero, el \textit{uid
(identificación de usuario)}. Esto es debido a que, para un sistema, un número es mas fácil
y rápido de procesar que un nombre. Una base de datos o tabla
asociada a dichos \texttt{uid}, por fuera del núcleo, asigna un nombre alfanumérico 
único a cada \texttt{uid}. Dicha base de datos también contiene información adicional, 
como su nombre completo, el intérprete de comandos predeterminado, etcétera. 

Para crear un usuario, se necesita agregar información sobre el mismo a la
base de datos y crear un directorio ``inicio'' (directorio principal
del usuario, en inglés ``home'') para él. También puede ser necesario educar al 
usuario, y configurar un ambiente conveniente para él.

La mayoría de las distribuciones de GNU/Linux cuentan con programas para crear
cuentas. Existen varios programas disponibles en los repositorios,
o en Internet \footnote{En Internet puede comenzar buscando en:
http://sourceforge.com y http://freshmeat.com}.  
Los comandos mas utilizados son \texttt{\textbf{adduser}} y
\texttt{\textbf{useradd}}. También existen herramientas gráficas, usualmente 
cada entorno de escritorio gráfico (como ser GNOME, KDE, etcétera) proveen  
implementaciones propias. Más allá de la herramienta particular que se utilice,  
el trabajo manual requerido para crear una nueva cuenta es sencillo. En la sección
\textit{Crear un usuario a mano} se describe todos los detalles relacionados.


\subsection{ \texttt{/etc/passwd} y \texttt{/etc/shadow}}

La base de datos básica de usuarios en un sistema Unix, o GNU/Linux en particular, 
es un archivo de texto  \texttt{/etc/passwd} (llamado el \textit{archivo de
contraseñas}\footnote{Este nombre se mantiene por compatibilidad, aún cuando 
en los sistemas modernos dicho archivo ya no contenga las contraseñas efectivamente.}), 
que lista todos los nombres de usuarios válidos y su información asociada. 

Este archivo es consultado por el sistema en cada intento de ingreso al mismo (login).
El archivo tiene una línea por usuario, y es dividido en siete campos delimitados por 
dos puntos ``:'' en el siguiente orden:


	\begin{enumerate}
	\item{Nombre de usuario: alfanumérico y sensible a mayúsculas y minúsculas.}
	\item{Contraseña, de modo cifrado: es opcional. Usualmente en este campo encontraremos 
	una ``x'', lo que indica el uso del archivo \texttt{/etc/shadow} como base de datos de 
	contraseñas}
	\item{Número de identificación de usuario (uid).} 
	\item{Número de identificación de grupo: se refiere al grupo primario del usuario. Puede 
	pertenecer a más grupos, lo cual se especifica en el archivo \texttt{/etc/group}}.
	\item{Nombre completo u otra información descriptiva de la cuenta.}
	\item{Directorio inicio (directorio principal del usuario, en inglés ``home'')}
	\item{Intérprete de comandos (programa a ejecutar al ingresar al sistema).}
	\end{enumerate}

El formato esta explicado con mas detalles en la página de manual correspondiente: 
\texttt{man 5 passwd} (sección 5 ``Formato de ficheros y convenios'').

\textit{Acerca del uso del segundo campo y el archivo \texttt{/etc/shadow}}: 

Cualquier usuario del sistema debe poder leer el archivo de \texttt{/etc/passwd}\footnote{
Observe los permisos asignados a este archivo.} ya que, 
entre otras, muchos programas ejecutados por el usuario requieren de este acceso 
para obtener información y funcionar correctamente. Esto significa que: si almacenamos 
la contraseña en el segundo campo, esta también estará disponible para todos. En principio, 
esto no parece ser un problema porque dichas contraseñas se encuentran cifradas.  
Sin embargo, dicho cifrado puede ser quebrado, sobre todo si la contraseña es ``débil'' (por ejemplo, 
una palabra de diccionario).  \footnote{Estadísticamente, según el estudio de
los métodos para romper claves cifradas, se ha establecido que aumenta
significativamente la seguridad una suma de características: tener más de seis
caracteres, combinar letras mayúsculas y minúsculas, a la vez que intercalar
también números.}. 

De lo anterior podemos concluir que tener las contraseñas almacenadas en 
en una base de datos (archivo en este caso) separada, 
no accesible por cualquier usuario, es una buena medida de seguridad. Es por este motivo 
que existe el archivo \texttt{/etc/shadow} en la gran mayoría, sino todas, los sistemas
GNU/Linux modernos.  

El archivo \texttt{/etc/shadow} es una alternativa en la manera de
almacenar las contraseñas: las claves cifradas se guardan en un archivo
en dicho archivo que solo puede ser leído y modificado por el administrador del 
sistema\footnote{Verifique los permisos en su instalación. En algunos casos se crea 
un grupo especial ``shadow'' al que se le permite leer dicho archivo.}. Así el 
archivo \texttt{/etc/passwd} solo contendrá un marcador especial en ese segundo campo. 
El formato de \texttt{/etc/shadow} está explicado con mas detalles en la página de 
manual correspondiente: \texttt{man 5 shadow} (sección 5 ``Formato de ficheros y convenios'').


\subsection{Elegir números de identificación de usuario y grupo}
En general dentro de muchos sistemas, no es importante cuáles es el número de 
identificación de usuario y grupo (uid y gid) más allá de los límites del sistema. 
Si bien suele reservarse el rango cero a cien 
para usuarios de sistema (por ejemplo: sys y daemon), esto es solo una convención, 
a excepción del uid cero que corresponde a root en cualquier caso.  

Sin embargo, cuando se utilizan sistemas sistemas de archivos de red, como 
por ejemplo NFS (Network file System), el número de identificación deber traspasar
los límites del sistema. Es decir, se necesitará que
los números de identificación de usuario (uid) y grupo (gid) sean los mismos en todos
los sistemas. Esto es porque el sistema de archivos de red también identifica al
usuarios (nombre de usuario) con su respectivo uid.  

Por lo anterior, si se desea usar un sistema de archivos de red NFS
\footnote{NFS: Network file System}, es necesario implementar un mecanismo 
para sincronizar la información de cada cuenta. Una alternativa es el sistema NIS
(del inglés ``Network Information Service''). La descripción de este servicio 
esta fuera del alcance de este curso. Sea cual sea la solución elegida, esta deberá 
garantizar la unicidad de nombres y uids entre los sistemas que comparten 
archivos a través de NFS u otros servicios del estilo. 


\subsection{Ambiente inicial: \texttt{/etc/skel}}
 	\footnote{ Apocope de la palabra inglesa
 	skeleton, que en castellano significa esqueleto, asiendo referencia al función
 de estructura.}

Cuando el directorio Inicio para un nuevo usuario es creado es
inicializado por medio del directorio \texttt{/etc/skel}. El
administrado del sistema puede crear archivos dentro de
\texttt{/etc/skel} que proveerán un amable entorno predeterminado
para los usuarios. Por ejemplo, el puede crear un
\texttt{/etc/skel/.profile} que configura  las variable de entorno
de algún editor mas amigable para los usuarios nuevos.


Como sea, usualmente lo mejor es conservar dicho directorio lo mas pequeño
que sea posible, ya que en el futuro será imposible actualizar los archivos de
los usuarios. Por ejemplo, si cambia el nombre de un editor a uno mas amigable,
todos los usuarios tendrán que editar su archivo  \texttt{.profile}.
El administrador del sistema podría tratar de hacer esto automáticamente con un
script \footnote{Lenguaje de programación cuyo código no necesita ser
compilado para ser ejecutado, por lo general es interpretado por el shell.
Ver también "Expresiones regulares".}, pero
	casi con seguridad resultará que se corrompa el archivo de
alguno.  Siempre que sea posible, es mejor poner lo que sea configuración global
dentro de archivos globales, como es /etc/profile. De esta manera es posible
actualizarlo sin corromper la configuración de ningún usuario.  






\subsection{Crear un usuario a mano}

Para crear una nueva cuenta a mano, sigue estos pasos:

\begin{itemize}
	
	\item{ Editar \texttt{/etc/passwd} con
	\texttt{\textbf{vipw}} y agregar una nueva linea por cada nueva cuenta.
	Teniendo cuidado con la sintaxis.  \textit{\bf No lo edite directamente
	con un editor!}. Utilice el comando \texttt{\textbf{vipw}}, el cual bloquea el
	archivo, así otros comandos no tratarán de actualizarlo al mismo tiempo.
	Debería hacer que el campo de la contraseña sea `\texttt{*}',
	de esta forma es imposible ingresar al sistema.}

	\item{ Similarmente, edite \texttt{/etc/group} con
	\texttt{\textbf{vigr}}, si necesita crear también un
	grupo.} \item{ Cree el directorio Inicio del
	usuario con el comando \texttt{\textbf{mkdir}}.}
	\item{ Copie los archivos de \texttt{/etc/skel} al
	nuevo directorio creado 
	\footnote{cp /etc/skel/* /ruta
(donde ruta será por convención /home/"nombre de usuario"}}
	\item{ Corrija
	la pertenencia del dueño y permisos con los comandos
	\texttt{\textbf{chown}} y \texttt{\textbf{chmod}} (Ver paginas de
	manual de los respectivos comandos). La opción \texttt{-R} es
	muy útil. Los permisos correctos varían un poco de un sitio a otro, pero
	generalmente los siguientes comandos harán lo correcto:


\begin{verbatim}
cd /home/nuevo-nombre-de-usuario
chown -R nombre-de-usuario.group .  
chmod -R go=u,go-w .  
chmod go= .
\end{verbatim}


	}
	
	\item{ Asigne una contraseña con el comando
	\texttt{\textbf{passwd}}}

\end{itemize} 
	
	Después de asignar la contraseña del usuario en el ultimo paso, la
	cuenta funcionara. No debería configurar esto hasta que todo lo demás
	este hecho, de otra manera el usuario puede inadvertidamente ingresar al
	sistema mientras copias los archivos de configuración de su entorno de
	trabajo.

A veces es necesario crear cuentas "falsas"
		
		\footnote{¿Usuarios Surrealistas?} que no son
		usadas por personas. Por ejemplo, para configurar un servidor
		FTP
		\footnote{FTP: File
		transfer Protocol.} anónimo (así cualquiera podrá acceder a los archivos por
		él, sin tener que conseguir una cuenta de usuario en el sistema
		primero) podría crear una cuenta llamada "ftp". En esos casos,
		usualmente no es necesario asignar una contraseña (el ultimo
		paso de arriba).  Verdaderamente, es mejor no hacerlo, para que
		nadie puede usar la cuenta, a menos que primero sea root/cuenta
		administrador, y así convertirse en cualquier usuario.







 \section{ Cambiar las propiedades del usuario}

\begin{itemize}
	\item Hay algunos comandos para cambiar varias propiedades de cualquier cuenta: \\ \\
\begin{tabular}{ l l }
	\texttt{chfn} & cambia el campo del nombre completo \\
	\texttt{chsh} & cambia el campo del interprete de comandos \\
	\texttt{passwd} & cambia la contraseña \\
\end{tabular}
\end{itemize}


Normalmente los usuarios solo pueden cambiar las propiedades de sus propias
cuentas. A veces es necesario deshabilitar estas posibilidades (por medio del
comando \texttt{\textbf{chmod}})  para los usuarios normales, por ejemplo en un
ambiente con muchos usuarios novatos.

Otras tareas pueden ser necesarias hacerlas manualmente. Por ejemplo,
cambiar el nombre de usuario, editando el archivo
\texttt{/etc/passwd} directamente (recuerda hacerlo con el
\texttt{\textbf{vigr}}). También para agregar o quitar a uno o varios usuarios
de uno o mas grupos, editando \texttt{/etc/group} (con
\texttt{\textbf{vigr}}). Este tipo de tareas tienden a ser mas raras, de todas
maneras, siempre hay que ir con cuidado: si cambia un nombre de usuario, dicho
usuario dejara de acceder a su cuenta de correo a menos que también le genere un
alias a su dirección de correo.
	
		\footnote{Los usuarios pueden cambiar su
		nombre por haberse casado, por ejemplo, y quieran tener su cuenta de usuario
		actualizada para reflejar su nuevo nombre.}
		









\section{Borrando usuarios}

Para borrar un usuario, primero borre los archivos que le pertenezcan,
casilla de correo, alias de correo, trabajos de impresión, trabajos pendientes a
través de los demonios \texttt{\textbf{cron}} y \texttt{\textbf{at}}, y
cualquier otra referencia al usuario.  Entonces quite las correspondientes
lineas relevantes de los archivos \texttt{/etc/passwd} y
\texttt{/etc/group}	 (recuerde borrar al usuario de todos los grupos
a los cuales pertenecía).  Puede ser buena idea deshabilitar la cuenta antes de
empezar a borrar cosas para prevenir que el usuario use la cuenta mientras esta
siendo eliminado.

Recuerde que los usuarios pueden tener archivos fuera de su directorio
Inicio.  Para encontrarlos use el comando:

 find / -user username 

Como sea, note que el comando find puede tomar \textit{\bf mucho tiempo} si
tiene discos muy grandes o si monta un disco de red.


Algunas distribuciones tiene comandos especiales para realizar esta tarea. Por 
ejemplo
\texttt{\textbf{deluser}} o \texttt{\textbf{userdel}}. Igualmente, es fácil
hacerlo a manualmente, y de todas maneras puede que el comando no lo haga todo.






\section{ Deshabilitar un usuario temporalmente}

A veces es necesario deshabilitar una cuenta temporalmente, sin borrarla.
Por ejemplo, un usuario pudo dejar de pagar sus cuentas, o el administrador de
sistema puede sospechar que un cracker
	\footnote{En este caso traducible a invitado inesperado y
	malicioso} tiene la contraseña de esa
cuenta.

La mejor manera de deshabilitar una cuenta es cambiar su intérprete de
comandos por otro programa que solo envía mensajes a la pantalla. De esta manera,
cualquiera sea la forma que intente entrar al sistema con esa cuenta fracasará,
y sabrá porqué. El mensaje puede decir que el usuario se contacte con el
administrador de sistema para que cualquier problema sea tratado con/por
él. Es posible también cambiar el nombre de usuario o contraseña
del mismo por otro, en tal caso el usuario no sabrá que pasa. Usuarios confusos
significa mas trabajo.  \footnote{Pero ellos pueden llegar a ser
\textit{muy} divertidos, si eres un BOFH.}

	
	

 Una manera simple de para crear un programa especial es escribir una
"cola de script": 

 
\texttt{\#!/usr/bin/tail +2 }
Esta cuenta ha sido cerrada por razones de seguridad.
Por favor llame al 555-1234 y espere que lleguen los hombres de negro.



Los primeros dos caracteres ('\texttt{\#!}') le dicen al núcleo que el
resto de la línea es un comando que necesita ejecutarse por medio de un
interprete. El comando \texttt{\textbf{tail}} en este caso manda una salida en
pantalla de todo excepto de la primera línea de la salida estándar.

El usuario billg
	\footnote{ billg: referencia a
	Bill Gate, uno de los cofundadores de la empresa Microsoft.} es sospechado
	de infringir la seguridad, el
administrador del sistema puede hacer algo como esto: 



\colorbox{grey}{\parbox[t]{0.95\linewidth}{ \vspace*{0.5cm} 
{\tt  
\# chsh -s /usr/local/lib/no-login/security billg \\
\# su - tester \\
 \\
    Esta cuenta ha sido cerrada por razones de seguridad. \\
    Por favor llame a 555-1234 y espere hasta que lleguen los Hombre de Negro  \\
 \\
\#  
} \vspace*{0.5cm} } } 



El propósito del comando \textbf{su} es verificar que el cambio
funciona, por supuesto.  

"Tail script" debe mantenerse en un directorio separado, así sus nombres
no interfieren con los comandos de los usuarios normales.




\end{document}
