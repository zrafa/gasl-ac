
\documentclass[12pt]{article}

\usepackage{apuntes-estilo}
\usepackage{fancyhdr,lastpage}
\usepackage{color,colortbl}



\def\maketitle{

% Titulo 
 \makeatletter
 {\color{bl} \centering \huge \sc \textbf{
Inicio y Apagado de un Sistema GNU/Linux \\ 
%  \large \vspace*{-8pt} \color{black} Vim el Editor de Seis Billones de Dólares
 \vspace*{8pt} }\par}
 \makeatother


% Autor
 \makeatletter
 {\centering \small 
 	Departamento de Ingeniería de Computadoras \\
 	Facultad de Informática - Universidad Nacional del Comahue \\
 	\vspace{20pt} }
 \makeatother

}

% Custom headers and footers
\fancyhf{} % clear all header and footer fields
\fancypagestyle{plain}{\fancyhf{}}
  	\pagestyle{fancy}
 	\lhead{\footnotesize Inicio y Apagado de un Sistema GNU/Linux - Departamento de Ingeniería de Computadoras}
 	\rhead{\footnotesize \thepage\ }	% "Page 1 of 2"

\def\ti#1#2{\texttt{#1} & #2 \\ }



\begin{document}

\thispagestyle{empty}
\maketitle
\setlength{\parindent}{0pt}


\begin{abstract}
 En esta sección se explica lo que sucede en un sistema GNU/Linux al momento
de encender o apagar la computadora, y de cómo debe hacerse apropiadamente. Si
no se sigue el procedimiento adecuado, los archivos se pueden dañar o perder.
\end{abstract}



\section{Una introducción al proceso de inicio y finalización del sistema}

El acto de  encender una computadora y cargar su sistema operativo 
se le llama en Inglés \textit{booting}. El nombre viene de
la idea del computador iniciándose a sí mismo sin ayuda. Si bien las particularidades
del proceso varían dependiendo del hardware que estemos contemplando, hay ciertas
fases generales que aplican cualquiera sea este. 

Cuando se enciende una computadora, esta ejecuta un código fijo almacenado en una 
memoria no volátil (que no pierde su contenido cuando el hardware no está energizado) en la 
placa base. Este código intentará deducir dónde se encuentra
la siguiente porción de software a ejecutar, que permitirá en última instancia 
cargar el sistema operativo y comenzar a ejecutarlo. Este código inicial es muy pequeño 
y sencillo por lo 
que usualmente es necesario software adicional para completar la carga del sistema. 

En este apunte en particular describiremos el proceso de arranque de un sistema 
GNU/Linux instalado en una computadora personal típica. La elección se debe a que 
este es el tipo de hardware con el que el lector se encontrará con mas frecuencia 
durante esta etapa.  

 Al finalizar un sistema Linux (shutdown), se le ordena a cada proceso
terminar. Esto causa que los archivos sean cerrados y que se realicen otras
tareas que sirven para mantener el sistema en orden. A continuación se desmontan
los sistemas de archivos y las áreas de intercambio\footnote{Sección de disco utilizada 
para implementar la memoria virtual del sistema}. Finalmente, se muestra un mensaje
en la consola indicando que el equipo puede apagarse o bien se apaga automáticamente. 
Si no se sigue el procedimiento anterior pueden y sucederán cosas terribles. Uno de los problemas 
que puede ocurrir por ejemplo, es que el buffer caché
\footnote{La lectura desde el disco es mas lenta en comparación con el acceso a 
memoria primaria o RAM. Además, es común leer la misma parte del disco varias 
veces durante periodos relativamente cortos de tiempo. Leyendo la información del
disco una sola vez y luego manteniéndola en la memoria hasta que no sea necesaria,
puede acelerar todas las lecturas posteriores con respecto a la primera. Esto es 
llamado ``buffering'' de disco (disk buffering), y la memoria usada para ese propósito 
es llamada buffer cache.}
 del sistema de archivos no sea escrito a
disco (o a su medio de almacenamiento correspondiente), lo que significa que los
datos contenidos en él se pierdan y que por lo tanto la información contenida en
el sistema de archivos del disco sea inconsistente y posiblemente inutilizable.

\section{Más acerca del arranque de GNU/Linux}

Al encender una computadora, en lo que se denomina un arranque en frío, esta se
comprueba a sí misma usando un código que se denomina POST (Power On Self
Test). El objetivo de dicha prueba es verificar que los elementos imprescindibles
para su funcionamiento están presentes y disponibles (la memoria RAM, por
ejemplo). Dados los escasos recursos con los que cuenta un ordenador
al que pudieran fallarle partes muy básicas, los fallos suelen ser notificados al
usuario por medio de indicaciones sonoras a través de un pequeño altavoz que
suele llevar la placa base, usando secuencias de pitidos predeterminadas para
cada tipo de fallo; en otros casos serán el parpadeo de ciertas luces indicadoras
lo que indiquen el tipo de fallo. 

Si la prueba POST se pasa con éxito, se ejecuta a continuación un programa
denominado ``bootstrap loader'' (cargador de inicio) que reside en la memoria de la
BIOS (del inglés ``Basic Input Output System''). El objetivo de este programa 
es encontrar un \textit{sector de arranque} en alguno de
los dispositivos de almacenamiento disponibles (antes tan sólo discos rígidos o
flexibles, pero hoy en día es posible arrancar de prácticamente cualquier dispositivo
de almacenamiento: llaves USB, tarjetas CF -Compact Flash- que pueda
montarse como disco, o incluso desde una red local).
Un \textit{sector de arranque} es el primer sector de un disco y contiene un pequeño
programa  que es capaz de cargar un Sistema Operativo. Al \textit{sector de arranque} 
se lo conoce  en inglés como \textit{boot sector} o
\textit{master boot record} (MBR o sector de arranque maestro). Si
el disco contiene varias particiones, cada una de ellas tiene su propio sector
de arranque (primer sector de cada partición).  

El cargador de inicio tiene una lista de lugares en las que buscar un sector de
arranque. Esto es lo que se programa cuando en la BIOS seleccionamos los dispositivos
de los que arrancar y su orden.

El pequeño programa que se ejecuta desde el sector de arranque puede ser de
varios tipos, y es lo suficientemente pequeño para que quepa en un único sector, 
cuya responsabilidad será la de leer el sistema operativo desde el disco, cargarlo 
en memoria y ponerlo en ejecución. En el caso de GNU/Linux, uno de los más famosos fue LILO
\footnote{Debido a su sencillez y eficiencia muchos administradores siguen 
optando por LILO}, sin embargo 
hace tiempo que muchas distribuciones utilizan de modo predeterminado un cargador alternativo,
llamado GRUB (mucho mas complejo pero a la vez más potente que LILO). Otros Sistemas Operativos 
tienen su propio programa cargador.

Usaremos LILO en la descripción, pues es más ilustrativo.
En el caso concreto de LILO, el sector de arranque carga una parte
de éste, denominada ``first stage boot loader'' (primer paso del cargador de inicio).
Su misión es cargar y ejecutar el segundo paso del cargador de inicio.
Esta segunda parte suele mostrar una selección de Sistemas Operativos a cargar,
procediendo luego a cargar el sistema seleccionado por el usuario (o bien el
que se haya configurado como sistema predeterminado, tras un tiempo de espera). 
Esta información está incluida dentro del cargador de inicio
y, para introducirla, se usa el comando \texttt{lilo} que a su vez usa el contenido del
archivo \texttt{/etc/lilo.conf}. Todo ello sucede, por supuesto, con el ordenador ya en marcha.

Una vez que LILO ha cargado el ``kernel'' (núcleo) Linux, le transfiere el control a éste.
El kernel Linux luego de que ha sido cargado en memoria y comienza a
ejecutarse, inicializa el hardware y los controladores de los dispositivos. 
Adicionalmente el kernel recibe algunos parámetros durante su carga. De éstos, 
uno de los más importantes es el que le dice al núcleo qué dispositivo usar 
como sistema de ficheros raíz, es decir, lo
que en UNIX se denomina ``/''. En una computadora personal típica, la raíz sería
una partición de un disco rígido. 

Si el núcleo ha conseguido montar el sistema de ficheros raíz, lo siguiente a
ejecutar es el programa ``\texttt{init}'' (para más detalle \texttt{man 8 init}). 
En general, ``\texttt{init}'' es sólo el programa que se encarga de arrancar el resto de
procesos que la máquina debe ejecutar. Entre sus tareas está el comprobar y
montar sistemas de archivos adicionales, así como iniciar programas servidores (daemons)
para cada función necesaria. Otra tarea importante es la de ejecutar procesos
``\texttt{getty}'' cuya misión es proporcionar consolas donde poder iniciar sesión en el
sistema. Las ordenes a seguir por \texttt{init} están en el fichero \texttt{/etc/inittab}. 
A partir de
ese punto, y en función del sistema de inicialización utilizado (el más frecuente es
el denominado ``System V'') el proceso seguido por \texttt{init} es distinto, pero en el
fondo obedece más a un factor de forma, es decir, a una estrategia de
ordenamiento de los ``scripts'' de inicialización de los distintos procesos que a un
factor de fondo.

Una vez iniciados todos los servicios y procesos de entrada de usuario, o bien
estamos delante de una consola de texto en la que el ordenador nos pide que nos
identifiquemos, o bien estamos ante una consola gráfica que nos pide lo mismo, o
bien estamos ante una pantalla llena de opciones sobre qué ejecutar (escuchar
música, ver películas, por ejemplo) si el sistema arranca bajo un usuario
predeterminado y no nos pide iniciar sesión explícitamente.

La siguiente es la salida por pantalla de una secuencia de arranque, en 
la que se observa el gestor de arranque LILO y algunas comprobaciones de 
hardware realizadas luego de la carga del kernel Linux: 

\colorbox{grey}{\parbox[t]{0.95\linewidth}{ \vspace*{0.5cm} {\tt 
LILO boot: \\
Loading linux. \\
Console: colour EGA+ 80x25, 8 virtual consoles \\
Serial driver version 3.94 with no serial options enabled \\
tty00 at 0x03f8 (irq = 4) is a 16450 \\
tty01 at 0x02f8 (irq = 3) is a 16450 \\
lp\_init: lp1 exists (0), using polling driver \\
Memory: 7332k/8192k available (300k kernel code, 384k reserved, 176k \\
data)
Floppy drive(s): fd0 is 1.44M, fd1 is 1.2M \\
Loopback device init \\
Warning WD8013 board not found at i/o = 280. \\
Math coprocessor using irq13 error reporting. \\
Partition check: \\
  hda: hda1 hda2 hda3 \\
VFS: Mounted root (ext filesystem). \\
Linux version 0.99.pl9-1 (root@haven) 05/01/93 14:12:20
 } \vspace*{0.5cm} } } 


El idioma y la forma exacta del texto es diferente en cada sistema, en función
del hardware instalado, la versión de Linux que se está utilizando y como se ha
establecido la configuración del sistema.



\section{ Más acerca del apagado (shutdown)}

 Es sumamente importante comprender y seguir los pasos correctos al
momento de finalizar un sistema GNU/Linux. Si no realiza este procedimiento, es 
probable que sus sistemas de archivos se vean perjudicados. Esto sucede porque Linux 
tiene un cache de disco que no escribe a disco cada vez que se le solicita, sino 
que solamente lo hace a intervalos. Esta manera de proceder mejora de manera 
significativa el desempeño del sistema, pero al mismo tiempo significa que si se 
apaga el equipo de imprevisto puede perder información que debería estar en 
sus sistemas de archivos (y obviamente en sus discos). Esto último sucede porque la cache puede
contener una gran cantidad de datos que se pierden con el apagado, y lo que está
en el disco puede no ser un sistema de archivos totalmente funcional (debido a
que solamente parte de la información ha sido transcrita de la cache al disco
duro).  

 Una razón adicional para no desconectar directamente el sistema de la
energía (presionando por ej. el botón de apagado) es que en un ambiente
multitarea y multiusuario (también conocido como tiempo compartido) existen 
diversos procesos que se están ejecutando en segundo plano, y
desconectar la computadora en este momento puede ser desastroso. Utilizando el
procedimiento correcto para el apagado del equipo garantiza que todos los
procesos en segundo plano puedan guardar sus datos.  

 El comando para finalizar correctamente un sistema GNU/Linux es
\texttt{\textbf{shutdown}}. Se utiliza generalmente de una de las 
siguiente maneras: 

 Si Ud. es el único usuario del sistema, entonces debe finalizar todos los
programas que estén en ejecución, finalizar todas las sesiones (log out) de
todas las consolas virtuales, e iniciar una sesión como usuario root, y 
ejecutar el comando \texttt{\textbf{shutdown -h now}}. Si
desea postergar durante algún lapso el comando shutdown, reemplace
\texttt{now} con un signo + (mas) y un número que indica minutos de
espera.  

 Alternativamente, si el sistema está siendo utilizado por muchos
usuarios, utilice el comando shutdown -h +time mensaje, donde time es el número
de minutos en que se posterga la detención del sistema, y el mensaje es una
explicación breve de el por qué se está apagando el sistema.

\colorbox{grey}{\parbox[t]{0.95\linewidth}{ \vspace*{0.5cm} 
{\tt
\# shutdown -h +10 'Instalaremos un nuevo disco. El sistema  \\
estará funcional en aproximadamente tres horas.' \\
\#
} \vspace*{0.5cm} } } 

El ejemplo advierte a todos los usuarios que el sistema se apagará en diez
minutos, y que sería mejor que se desconectaran o se arriesgan a perder la
información con la que están trabajando. La advertencia se muestra en cada
terminal donde existe un usuario conectado, incluyendo los emuladores de
terminales para el sistema X Window. 



\colorbox{grey}{\parbox[t]{0.95\linewidth}{ \vspace*{0.5cm} 
{\tt  Broadcast message from root (ttyp0) Wed Aug  2 01:03:25 1995... \\
Instalaremos un nuevo disco. El sistema estará funcional \\
en aproximadamente tres horas.  The system is going DOWN for system halt \\
in 10 minutes !!   } \vspace*{0.5cm} } } 


El mensaje de advertencia (warning)
se repite automáticamente varias veces antes de pagar la máquina con
intervalos cada vez más frecuentes.

 Después del tiempo de postergación, cuando el proceso de apagado real
empieza, se desmontan todos los sistemas de archivos (excepto el sistema de
archivos raíz /), se finalizan (kill) los procesos de los usuarios (si existen
aún usuarios dentro del sistema), los demonios y finalmente se desmonta el
sistema de archivos raíz. Cuando todo este proceso finaliza,
\texttt{\textbf{init}} muestra un mensaje indicando que se puede apagar la
máquina. Es entonces y sólo entonces que es posible bajar el switch o
interruptor de suministro eléctrico, si es que no se hace automáticamente.

 Algunas veces, aunque es raro en un buen sistema, es imposible concluir
el sistema de forma adecuada. Por ejemplo, si ocurre un error fatal con el
kernel, puede ser imposible ejecutar cualquier comando nuevo, haciendo la
finalización normal del sistema inviable. Todo lo que se puede hacer en este
caso es esperar que ningún daño severo ocurra y entonces desconectar la máquina.


\section{ Reinicio (Rebooting)}

 Rebooting significa iniciar el sistema nuevamente. Se puede realizar
finalizando el sistema, apagando la máquina y entonces encendiendo nuevamente. Un
método más práctico para tener el mismo efecto consiste en indicarle al comando
\texttt{\textbf{shutdown}} que reinicie el sistema, en vez de que lo detenga.
La opción \texttt{-r} del comando \texttt{\textbf{shutdown}} puede
realizar la acción anterior, y puede utilizarse inmediatamente, por ejemplo:
\texttt{\textbf{shutdown -r now}}.  

 La mayoría de los sistemas GNU/Linux ejecutan el comando \texttt{\textbf{shutdown -r
now}} cuando se oprime la combinación de teclas \texttt{ctrl-alt-del} (y
reinician el equipo). La manera de operar de \texttt{ctrl-alt-del} se puede configurar y
es una buena idea postergar la acción durante algún tiempo antes de reiniciar
una máquina multiusuario. Los sistemas en los que cualquier persona puede
acceder físicamente es conveniente configurar \texttt{ctrl-alt-del} para que no haga
nada.

En algunos Linux también logramos el mismo efecto con el comando reboot.

\section {Niveles de ejecución}


Luego de que de que el proceso \texttt{init} es ejecutado por el kernel, comienza 
la ejecución de servicios y procesos que resultarán finalmente en un sistema 
completamente funcional. Este proceso también involucra una serie de etapas bien 
definidas, es un proceso escalonado, no inmediato. Por ejemplo para que un servicio de página web 
este disponible para su uso, es necesario que previamente 
se haya configurado la red. Lo mismo sucede durante el apagado, antes de 
eliminar el servicio de red, es conveniente que las aplicaciones que 
utilizan dicho servicio finalicen. Estas etapas definen lo que se conoce como 
\textit{\textbf{niveles de ejecución}}.

Un nivel de ejecución es una configuración de software del sistema que permite 
existir sólo a un grupo de procesos seleccionado. Los procesos levantados por 
\texttt{init} para cada uno de estos niveles de ejecución se definen en el archivo 
\texttt{/etc/inittab}. Para más detalle lea la página del manual en línea 
correspondiente a inittab (\texttt{man inittab}).
 
Un nivel de ejecución es un estado de \texttt{init} y de la totalidad del 
sistema, y define qué servicios están operando en un momento particular. Los 
niveles de ejecución son identificados por números (observe la tabla a continuación). 

\definecolor{tcA}{rgb}{1,0.92549,0.678431}
\begin{center}
\begin{tabular}{|c|c|}\hline
% use packages: color,colortbl
\rowcolor{tcA}
\textbf{Número de nivel} & \textbf{Función del nivel}\\\hline
0 & Parar el sistema\\\hline
1 & Modo de usuario individual (para tareas especiales de administración).\\\hline
2-5 & Operación normal (definidas para los usuarios).\\\hline
6 & Reiniciar el sistema.\\\hline
\end{tabular}
\end{center}

No existe aún un consenso de cómo utilizar los niveles de ejecución definidos 
para usuarios (2 a 5). Algunos administradores  
de sistemas utilizan estos niveles de ejecución para definir cuales subsistemas están trabajando, 
por ejemplo, podría definirse que el nivel de ejecución cinco este asociado al entorno gráfico, 
de modo que cuando se pide al sistema iniciar la transición hacia el nivel cinco, el sistema 
iniciará los servicios requeridos para iniciar el entorno gráfico (X, gdm, gnome, etc.). Esta 
estrategia evita tener que realizar estas tareas manualmente cada vez que, por ejemplo 
se reinicia la computadora.  

Los comandos \texttt{runlevel} o \texttt{who -r}, entre otros, nos permiten 
identificar en qué nivel de ejecución se encuentra actualmente el sistema. 
De este modo un administrador de sistemas puede saber cuáles son los servicios
que debieran estar operativos. 

Los detalles sobre el modo particular en que cada distribución GNU/Linux, 
asocia y ejecuta los servicios en cada nivel de ejecución, está fuera 
del alcance de este curso. Sólo mencionaremos que, en la actualidad 
la mayoría de las distribuciones utiliza lo que se conoce como
\textit{\textbf{System V init scripts}} que básicamente consiste en 
un conjunto de scripts asociados a cada nivel de ejecución, que se ejecutan 
en un orden particular cada vez que el sistema inicia una transición 
hacia ese nivel de ejecución (por ejemplo al invocar el comando shutdown 
o al encender la computadora). También existen otros esquemas como 
\textit{\textbf{Systemd}} desarrollado por freedesktop.org y utilizado por
muchas distribuciones en sus versiones más recientes, como ser Debian, Ubuntu, 
Suse y Fedora entre otras. 

En particular \textit{Systemd} reemplaza el concepto de ``nivel de ejecución'' por 
``target unit''. Nace debido a la necesidad de agilizar el arranque del sistema (en 
cierta medida por la explosión de equipos móviles que corren bajo Linux); la 
especificación de dependencias más complejas entre servicios, entre otras razones.  
Esta fuera del alance de este curso un análisis completo de Systemd, para más 
información referirse a freedesktop.org. 

\subsection{ Modo usuario individual (single user mode)}

 El comando \texttt{\textbf{shutdown}} también se puede utilizar para
cambiar el nivel de ejecución del sistema. En particular nos interesa el 
cambio al nivel de modo de usuario individual
(single user), en el cual nadie puede iniciar una sesión de usuario, pero root
puede utilizar la consola. Es útil para efectuar tareas de administración del
sistema que no se pueden realizar cuando el sistema se ejecuta normalmente.

 El monousuario o usuario individual garantiza al administrador la existencia 
de un conjunto de servicios mínimos para realizar tareas administrativas sin 
la interferencia de servicios no esenciales. Por ejemplo, podría definirse que 
en este nivel de ejecución, los servicios de red (y por ende todos los servicios
que lo utilizan) se encuentren apagados, de modo que la única forma de 
acceso sea a través de la consola. 

\section*{Licencia}
This is a derived version of "Guía Para Administradores de Sistemas GNU/Linux" that 
can be found at \\ \texttt{http://www.ibiblio.org/pub/Linux/docs/LDP/system-admin-guide/translations/es/gasl.txt}.

Copyright (C) 1993-1998 Lars Wirzenius.

Copyright (C) 1998-2001 Joanna Oja.

Copyright (C) 2001-2003 Stephen Stafford.

Copyright (C) 2003-2004 Stephen Stafford and Alex Weeks.

Copyright (C) 2004-Present Alex Weeks.

Copyright (C) 2013-Present Rafael Ignacio Zurita

Trademarks are owned by their owners.

Permission is granted to copy, distribute and/or modify this document under
the terms of the GNU Free Documentation License, Version 1.2 or any later
version published by the Free Software Foundation; with no Invariant
Sections, no Front-Cover Texts, and no Back-Cover Texts. A copy of the
license is included in the section entitled "GNU Free Documentation License".

\subsection*{Traducción de la licencia}
Esta es una obra derivada de "Guía Para Administradores de Sistemas GNU/Linux" que puede 
econtrarse en \\ \texttt{http://www.ibiblio.org/pub/Linux/docs/LDP/system-admin-guide/translations/es/gasl.txt}.

Las marcas registradas son propiedad de sus dueños.

Se concede autorización para copiar, distribuir y/o modificar este documento
bajo los términos de la Licencia de documentación libre GNU (FDL), Versión 1.2; 
sin Secciones Invariantes, sin textos de Portada, y sin Textos de Contra Portada. 
Una copia de la licencia se encuentra incluída en la sección "GNU Free Documentation License".


\end{document}
