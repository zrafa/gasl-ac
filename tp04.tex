%%% LaTeX Template: Article/Thesis/etc. with colored headings and special fonts
%%%
%%% Source: http://www.howtotex.com/

\documentclass[12pt]{article}


\usepackage{apuntes-estilo}
\usepackage{fancyhdr,lastpage}
\usepackage{verbatim}



\def\maketitle{

% Titulo 
 \makeatletter
 {\color{bl} \centering \huge \sc \textbf{
Trabajo práctico N 4 \\
\large \vspace*{-8pt} \color{black} Administrando Cuentas de Usuario
 \vspace*{8pt} }\par}
 \makeatother


% Autor
 \makeatletter
 {\centering \small 
	Introducción a la Administración de Sistemas \\
 	Departamento de Ingeniería de Computadoras \\
 	Facultad de Informática - Universidad Nacional del Comahue \\
 	\vspace{20pt} }
 \makeatother

}

% Custom headers and footers
\fancyhf{} % clear all header and footer fields
\fancypagestyle{plain}{\fancyhf{}}
  	\pagestyle{fancy}
 	\lhead{\footnotesize TP N 4 - Administrando Cuentas de Usuario}
 	\rhead{\footnotesize \thepage\ }	% "Page 1 of 2"

\def\ti#1#2{\texttt{#1} & #2 \\ }



\begin{document}

\thispagestyle{empty}
\maketitle
\setlength{\parindent}{0pt}

\paragraph{Los siguientes ejercicios se realizan en su totalidad en el equipo asignado al grupo.
Utilice la redirección y el editor \texttt{vi} para guardar las salidas y editar los resultados 
en un archivo llamado trabajo-practico-4.txt . En el archivo de resolución indique su nombre y apellido, y el número de ejercicio a resolver, junto a la salida incluyendo el comando que ejecuta. }

\section{Ejercicio 1.}


El comando  \texttt{adduser} se utiliza para crear nuevas cuentas de usuario de manera interactiva.

Su utilizacion es sencilla. Solo es necesario, como mínimo, especificar el nombre de usuario. Ejemplo : \texttt{adduser jgonzalez}

El sistema luego, le pedirá, de manera interactiva, información extra de la nueva cuenta a crear (el nombre completo real, la clave inicial, etc).

Utilizando el comando \texttt{adduser} agregue al sistema estos cinco usuarios nuevos :

Nombre de usuario : director \\
Nombre real : Juan Jose Gonzalez \\
Clave inicial : director01

Nombre de usuario : secretaria \\
Nombre real : Susana Gimenez \\
Clave inicial : secretaria02

Nombre de usuario : contaduria \\
Nombre real : Contador Esteban Gonzalez \\
Clave inicial : contador03

Nombre de usuario : legales \\
Nombre real : Abogado Mirko Garcia  \\
Clave inicial : legales04

Nombre de usuario : buffet \\
Nombre real : Mirta Legrant de Gonzalez \\
Clave inicial : buffet01


\section{Ejercicio 2.}
\begin{itemize}
\item - Utilice \texttt{cat} para visualizar el contenido de \texttt{/etc/passwd}, en la pantalla.
\item - Utilice \texttt{cat} y redirija la salida al archivo en donde coloca los resultados del practico.
\item - ¿Cuántos usuarios hay definidos en el sistema?
Ayuda: Utilice el comando \texttt{cat} redirigiendo la salida a \texttt{wc} (word count) para contar cuantas lineas
existen en el archivo.
\end{itemize}


\section{Ejercicio 3.}
Otra manera muy útil de verificar usuarios es a través de sesiones de login al sistema.

Verifique los cinco usuarios creados realizando conexiones secure shell (ssh) al equipo asignado al grupo.
Utilice el nombre de usuario y su clave inicial para corroborar que puede ingresar al sistema.

¿Cuál es el directorio HOME de cada usuario?



\section{Ejercicio 4.}
Realice una conexion al sistema con el usuario director.
Y responda :

\begin{itemize}
\item ¿Cuál es el directorio HOME del usuario director? ¿Que variable del ambiente del usuario reporta esa información?
\item ¿Con qué comando verifica todas las variables del ambiente del usuario director?

\item Intente crear un nuevo usuario utilizando el usuario director.
¿Es posible crear un nuevo usuario con director? Justifique porqué.
\end{itemize}



\section{Ejercicio 4.}
El comando \texttt{addgroup} funciona de manera similar a adduser, pero se utiliza para crear nuevos grupos de usuario.
Ejemplo de uso : \texttt{addgroup empleados} (crea en el sistema un grupo llamado "empleados").


Utilice \texttt{addgroup} para crear tres grupos nuevos en el sistema:

\begin{itemize}
\item directorio
\item empleados
\item todos
\end{itemize}

\section{Ejercicio 5.}
El comando \texttt{usermod} puede ser util para agregar usuarios a grupos, o modificar algunas opciones del usuario.
Por ejemplo, para agregar el usuario "director" al grupo "todos", puede ejecutar \texttt{usermod -G todos -a director}

Verifique con el manual del comando \texttt{usermod} que hacen las opciones -G y -a.

Modifique los grupos utilizando \texttt{usermod} para agregar usuarios a los grupos. De la siguiente manera :
 
\begin{itemize}
\item El grupo directorio deberia estar compuesto por los usuario director, legales, contaduria.
\item El grupo empleados deberia estar compuesto por secretaria, contaduria, legales, buffet.
\item El grupo todos debería estar compuesto por director, legales, contaduria.
\end{itemize}

\section{Ejercicio 6.}
Los comando \texttt{vipw} y \texttt{vigr} permite editar la base de datos de usuarios y grupos manualmente con vi.

Utilice el comando \texttt{vigr} para editar los grupos con vi. Agregue manualmente los usuarios buffet y secretaria al grupo "todos".


\section{Ejercicio 7.}

\begin{itemize}
\item Utilice el comando \texttt{deluser} para quitar del sistema a los usuarios buffet y legales.
\item Observe el archivo \texttt{/etc/passwd} para verificar que los usuarios han sido eliminados.
\item Listar los usuarios utilizando \texttt{cat} y el archivo \texttt{/etc/passwd}
\item Utilice los comando \texttt{cat} y \texttt{wc} para obtener la cantidad de usuarios del sistema.
\item Utilice los comando \texttt{cat} y \texttt{sort} para generar el mismo listado, pero en orden alfabetico.
\item Utilice el comando \texttt{passwd} para cambiar las claves iniciales del usuario director y secretaria. Verifique que la clave cambió realizando conexiones remotas con esos usuarios.
\end{itemize}

\section{Ejercicio 8.}
Repaso sistema de archivos.

\begin{itemize}
\item Listar los directorios \texttt{/bin} y \texttt{/usr/bin} por pantalla. ¿Qué tipo de archivos existen en esos directorios? (Ejemplo: Binarios, ejecutables, de texto plano, de configuracion, variables, temporarios, de documentacion. Un directorio puede tener mas de un tipo mencionado, por ejemplo "de texto y de configuracion").
\item Listar los directorios \texttt{/etc} por pantalla. Visualizar con \texttt{cat} de algun archivo a eleccion. ¿Qué tipo de archivos existen en esos directorios?
\item Listar el directorio \texttt{/home}. ¿Que tipo de información se guarda en ese directorio?
\item Listar el directorio \texttt{/tmp}. ¿Para qué sirve ese directorio?
\item Ejecutar \texttt{find /var}, y utilizar en conjunción con \texttt{wc} para conocer la cantidad de archivos que existe en el directorio \texttt{/var}.
\end{itemize}











\end{document}
