%%% LaTeX Template: Article/Thesis/etc. with colored headings and special fonts
%%%
%%% Source: http://www.howtotex.com/

\documentclass[12pt]{article}


\usepackage{apuntes-estilo}
\usepackage{fancyhdr,lastpage}
\usepackage{color,colortbl}
\usepackage{verbatim}

\def\maketitle{

% Titulo 
 \makeatletter
 {\color{bl} \centering \huge \sc \textbf{
 Monitoreo de recursos \\ 
\large \vspace*{-8pt} \color{black} Guía básica de monitoreo de recursos. 
 \vspace*{8pt} }\par}
 \makeatother


% Autor
 \makeatletter
 {\centering \small 
 	Departamento de Ingeniería de Computadoras \\
 	Facultad de Informática - Universidad Nacional del Comahue \\
 	\vspace{20pt} }
 \makeatother

}

% Custom headers and footers
\fancyhf{} % clear all header and footer fields
\fancypagestyle{plain}{\fancyhf{}}
  	\pagestyle{fancy}
 	\lhead{\footnotesize Reconocimiento y monitoreo de recursos - Departamento de Ingeniería de Computadoras}
 	\rhead{\footnotesize \thepage\ }	% ''Page 1 of 2''

\def\ti#1#2{\texttt{#1} & #2 \\ }



\begin{document}

\thispagestyle{empty}
\maketitle
\setlength{\parindent}{0pt}

\section*{Introducción}

Durante el tutorial anterior hablamos de reconocimiento de recursos físicos, identificar 
sus características y representación dentro del sistema operativo. Durante este texto nos dedicaremos a observar el consumo de esos recursos por parte de los procesos en ejecución. 

Monitorear un recurso significa verificar su estado, es decir si esta operando
dentro de los parámetros normales. En algunos casos, esto se traduce 
solamente en saber si el recurso está funcionando o no, mientras que en otros
la definición de operación normal puede ser menos clara. Por ejemplo en el 
caso de un CPU, además de si está presente o no, nos interesará saber cuál es 
su carga de procesos en ejecución. 
 

\section*{Monitoreando recursos}




\subsection*{Monitoreo de CPU}


\fcolorbox{black}{grey}{
\parbox[t]{1.0\linewidth}{ \vspace*{0.4cm}
{\bf Lo importante :} Incorporar el hábito de consultar los archivos de log e
identificar la información de interés para el recurso a identificar, o bien la ausencia de la misma.
\vspace*{0.4cm} } }

\colorbox{grey}{\parbox[t]{0.95\linewidth}{ \vspace*{0.5cm} { 
{\bf Ejemplo : Utilizando lscpu }
\\ \\
{\tt \small
\# lscpu \\
Architecture:          i686\\
CPU op-mode(s):        32-bit, 64-bit\\
Byte Order:            Little Endian\\
CPU(s):                2\\
On-line CPU(s) list:   0,1\\
Thread(s) per core:    1\\
Core(s) per socket:    2\\
Socket(s):             1\\
Vendor ID:             AuthenticAMD\\
CPU family:            16\\
Model:                 6\\
Stepping:              3\\
CPU MHz:               800.000\\
BogoMIPS:              5586.01\\
Virtualization:        AMD-V\\
L1d cache:             64K\\
L1i cache:             64K\\
L2 cache:              1024K\\
}
} \vspace*{0.5cm} } } 


\section*{Referencias}

[Intro] Tutuorial Introducción a la Administración de Sistemas GNU/Linux. Materia Introducción
a la Administración de Sistemas. UNCOMA. TUASSL. 

\section*{Licencia}

Este texto fue creado por Miriam Tamara Lechner y se encuentra bajo 
Licencia Creative Commons Atribución-CompartirDerivadasIgual 3.0 Unported

\end{document}
