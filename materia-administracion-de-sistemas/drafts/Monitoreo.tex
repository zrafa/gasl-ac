%%% LaTeX Template: Article/Thesis/etc. with colored headings and special fonts
%%%
%%% Source: http://www.howtotex.com/

\documentclass[12pt]{article}


\usepackage{apuntes-estilo}
\usepackage{fancyhdr,lastpage}
\usepackage{color,colortbl}
\usepackage{verbatim}

\def\maketitle{

% Titulo 
 \makeatletter
 {\color{bl} \centering \huge \sc \textbf{
 Monitoreo de recursos \\ 
\large \vspace*{-8pt} \color{black} Guía básica de monitoreo de recursos. 
 \vspace*{8pt} }\par}
 \makeatother


% Autor
 \makeatletter
 {\centering \small 
 	Departamento de Ingeniería de Computadoras \\
 	Facultad de Informática - Universidad Nacional del Comahue \\
 	\vspace{20pt} }
 \makeatother

}

% Custom headers and footers
\fancyhf{} % clear all header and footer fields
\fancypagestyle{plain}{\fancyhf{}}
  	\pagestyle{fancy}
 	\lhead{\footnotesize Reconocimiento y monitoreo de recursos - Departamento de Ingeniería de Computadoras}
 	\rhead{\footnotesize \thepage\ }	% ''Page 1 of 2''

\def\ti#1#2{\texttt{#1} & #2 \\ }



\begin{document}

\thispagestyle{empty}
\maketitle
\setlength{\parindent}{0pt}

\section*{Introducción}

Hemos hablado durante un tutorial anterior acerca de reconocimiento de recursos 
físicos, identificar sus características y representación dentro del 
sistema operativo.

Durante este texto nos dedicaremos a observar el consumo de esos recursos 
por parte de los procesos en ejecución. 

Monitorear un recurso significa verificar su estado, es decir si esta 
operando dentro de los parámetros normales. En algunos casos, esto se 
traduce solamente en saber si el recurso está funcionando o no, mientras 
que en otros la definición de operación normal puede ser menos clara. 
Por ejemplo en el caso de un CPU, además de si está presente o no, nos 
interesará saber cuál es su carga de procesos en ejecución. 


\section*{Procesos}

Un proceso es un programa en ejecución. Son digamos, las entidades vivas dentro 
de nuestro sistema. Cada programa que ejecutamos, por ejemplo el navegador web,
 puede disparar la ejecución de uno o mas procesos para llevar a cabo su tarea. A su vez, existen procesos que no son disparados por los usuarios, sino que 
los ejecuta el sistema operativo en sí mismo para cumplir sus funciones. Por 
ejemplo, el proceso \texttt{init}, ó el servicio de ssh. 

Los procesos en ejecución necesitan consumir recursos computacionales para 
llevar a cabo su función: CPU, memoria principal (RAM), memoria secundaria, 
transferir datos por red, etc. Algunos procesos necesitarán más tiempo de CPU, 
como puede ser un juego, o un programa de cáluclo; mientras que otros, por 
ejemplo el web browser, necesitarán más memoria principal (RAM), etc.

Es la intención de este apunte observar el consumo de recursos por parte de 
los procesos en ejecución. 
 
\subsection*{Estado de los procesos}

Los procesos no siempre se encuentran en ejecución. Como bien sabemos, la 
cantidad de CPUs disponibles es finita, por ende los procesos compiten por 
pequeñas porciones de tiempo de CPU. Momento en el cual, el proceso en 
ejecución tendrá la oportunidad de ejecutar algunas instrucciones de su 
código, hasta en tanto se acabe su porción de tiempo de CPU o bien ocurra 
algún otro evento que evite que el proceso se siga ejecutando (por ejemplo, 
porque está esperando recibir datos de la red). 

Si un proceso necesita abrir un archivo que se encuentra en 
memoria secundaria (disco), habrá una pequeña cantidad de tiempo en la que 
dicho proceso estará esperando por la respuesta del disco, que es muchas 
veces más lenta que la velocidad de procesamiento del CPU. Por lo que, mientras
el proceso espera, otros procesos pueden aprovechar y ejecutarse en el CPU. 
  
En un sistema GNU/Linux observaremos básicamente cinco estados en los que 
puede encontrarse un proceso: 
\begin{itemize}
\item R = Corriendo (running): significa que el proceso se encuentra en 
condiciones de ejecutarse en CPU o bien que está siendo ejecutado.  
\item S = Durmiendo (sleeping): el proceso se encuentra esperando por 
agluna señal o evento, no se encuentra en condiciones de ser ejecutado en 
CPU al momento. 
\item T = Trazado o detenido (traced or stopped): proceso que ha recibido 
la orden de detenerse (por ejemplo a pedido del usuario con el comando 
kill y la señal SIGSTOP). 
\item D = Durmiendo no interrumpible (uninterruptible sleep): se encuentra
esperando por un evento y no puede ser interrumpido. 
\item Z = zombie: proceso que termino y no fue reclamado por su proceso 
padre. 
\end{itemize}

Las trancisiones entre estados siguen el siguiente esquema: 
\begin{center}
 \includegraphics{process-states-s.jpg}
\end{center}


\subsection*{Utilización del CPU}

El CPU es utilizado por cada proceso por períodos cortos de tiempo. Dando 
al usuario, la senación de que todos los procesos corren simultáneamente.
El punto de interés en este caso, es averiguar cuán cargado se encuentra
el o los procesadores de una computadora en un momento particular. El 
primer comando que observaremos es \texttt{uptime}:


\colorbox{grey}{\parbox[t]{0.95\linewidth}{ \vspace*{0.5cm} { 
{\bf Ejemplo de salida de \texttt{uptime}:} \\
{\tt
# uptime \\
 22:28:41 up 1 day, 22:13, 10 users,  load average: 0,56, 0,65, 0,68
}
} \vspace*{0.5cm} } } 

Este comando muestra la hora actual, el tiempo que hace que el sistema se 
ha iniciado, la cantidad de sesiones iniciadas por usuarios, y la carga 
promedio del sistema en los últimos 1, 5 y 15 minutos. Estos últimos valores
son los de interés en esta sección. 

La carga promedio (load average) representa el número promedio de procesos
que se encuentran ya sea en estado ``ejecutable'' o ``ininterrumpible'' 
(estado ``D'').  Estado ``ejecutable'' puede ser tanto un proceso que se 
encuentra efectivamente ejecutándose en CPU o en cola de ejecución 
(listo para ser ejectudado en CPU).  Se debe tener en cuenta que el valor 
mostrado no se encuentra normalizado por el número de CPUs presentes en 
el sistema, es por esto que un load average de 1 (uno) en un sistema con 
un solo CPU significa que el mismo está cargado todo el tiempo, mientras 
que en un sistema con cuatro CPUs el mismo valor indica que estuvo oscioso
el 75\% del tiempo. 


\begin{center}
 \includegraphics{top.jpg}
\end{center}



\fcolorbox{black}{grey}{
\parbox[t]{1.0\linewidth}{ \vspace*{0.4cm}
{\bf Lo importante :} Incorporar el hábito de consultar los archivos de log e
identificar la información de interés para el recurso a identificar, o bien la ausencia de la misma.
\vspace*{0.4cm} } }

\colorbox{grey}{\parbox[t]{0.95\linewidth}{ \vspace*{0.5cm} { 
{\bf Ejemplo : Utilizando lscpu }
\\ \\
{\tt \small
\# lscpu \\
Architecture:          i686\\
CPU op-mode(s):        32-bit, 64-bit\\
Byte Order:            Little Endian\\
CPU(s):                2\\
On-line CPU(s) list:   0,1\\
Thread(s) per core:    1\\
Core(s) per socket:    2\\
Socket(s):             1\\
Vendor ID:             AuthenticAMD\\
CPU family:            16\\
Model:                 6\\
Stepping:              3\\
CPU MHz:               800.000\\
BogoMIPS:              5586.01\\
Virtualization:        AMD-V\\
L1d cache:             64K\\
L1i cache:             64K\\
L2 cache:              1024K\\
}
} \vspace*{0.5cm} } } 


\section*{Referencias}

[Intro] Tutuorial Introducción a la Administración de Sistemas GNU/Linux. Materia Introducción
a la Administración de Sistemas. UNCOMA. TUASSL. 

http://www.linuxjournal.com/article/9001


\section*{Licencia}

Este texto fue creado por Miriam Tamara Lechner y se encuentra bajo 
Licencia Creative Commons Atribución-CompartirDerivadasIgual 3.0 Unported

\end{document}
