%%% LaTeX Template: Article/Thesis/etc. with colored headings and special fonts
%%%
%%% Source: http://www.howtotex.com/

\documentclass[12pt]{article}


\usepackage{apuntes-estilo}
\usepackage{fancyhdr,lastpage}
\usepackage{color,colortbl}
\usepackage{verbatim}

\def\maketitle{

% Titulo 
 \makeatletter
 {\color{bl} \centering \huge \sc \textbf{
  Instalación, configuración y actualización de software.\\ 
\large \vspace*{-8pt} \color{black} Guía básica de administración de procesos. 
 \vspace*{8pt} }\par}
 \makeatother

% Autor
\makeatletter
 {\centering \small 
 	Departamento de Ingeniería de Computadoras \\
 	Facultad de Informática - Universidad Nacional del Comahue \\
 	\vspace{20pt} }
 \makeatother

}

% Custom headers and footers
\fancyhf{} % clear all header and footer fields
\fancypagestyle{plain}{\fancyhf{}}
  	\pagestyle{fancy}
 	\lhead{\footnotesize Administración de software - Departamento de Ingeniería de Computadoras}
 	\rhead{\footnotesize \thepage\ }	% ''Page 1 of 2''

\def\ti#1#2{\texttt{#1} & #2 \\ }



\begin{document}

\thispagestyle{empty}
\maketitle
\setlength{\parindent}{0pt}

\section*{Introducción}

Si bien la instalación, actualización y remoción de software hoy en día
puede parecer trivial, su mala administración suele ser una fuente de 
errores complejos que pueden derivar en daños graves al sistema. 
En particular en los sistemas de tipo UNIX como GNU/Linux el ecosistema
de aplicaciones es delicado, y para nada anárquico. Dicho ecosistema tiene 
una estructura y una forma particular de manipularlo. 

UNIX y variantes de sistemas GNU/Linux utilizan, generalmente, algún
sistema de empaquetado para facilitar la tarea de la administración de software.
Los {\bf paquetes} se han utilizado tradicionalmente para distribuir programas,
pero pueden ser también utilizados para contener archivos de configuración
o datos compartidos utilizados por varios paquetes. Su objetivo principal
es tratar de lograr un proceso de instalación tan atómico como sea posible,
de manera que si ocurre un error en el proceso, el administrador
sólo debe pausar la operación en curso (instalación, desistalación, etc)
y el sistema no quede en algún estado inconsistente.

Existen diversas maneras de gestionar el software en una distribución 
GNU/Linux. La forma más común y recomendada para un administrador es a través de paquetes
provistos por la organización que desarrolla la distribución GNU/Linux.
Pero también existen otros mecanismos
para la instalación de software en sistemas GNU/Linux :

- En forma de código fuente, en donde se debe compilar e instalar mediante herramientas de desarrollo
- En forma binaria provista por terceras partes
- En forma de paquetes provisto por terceras partes

Estos mecanismos y otros no listados quedan fuera del alcance de este artículo.

\section*{Administrando Paquetes de Software}

Las tareás básicas en la Administración de Software son :
\begin{itemize}
\item Actualizar la lista de paquetes disponibles;
\item Instalar, reinstalar, actualizar, y eliminar paquetes de software;
\item Obtener información acerca de los paquetes, incluyendo la versión, estado, dependencias, tamaño, integridad, etc;
\item Determinar qué archivos proporciona el paquete, y descubrir cual de los paquetes contiene un archivo determinado.
\end{itemize}

Existen una serie de conceptos que la 
mayoría de las distribuciones implementan y que un administrador de 
sistemas debe conocer con certeza aún cuando no conozca la implementación
particular de cada distribución.  Durante este apunte de desarrollan
 conceptos como paquete de software, 
repositorios, gestores de paquetes, entre otros que hacen a la 
administración de software dentro de las distribuciones GNU/Linux.


\subsection*{Paquete de software}


Un paquete de software es una serie de programas que se distribuyen conjuntamente. Algunas de las razones suelen ser que el funcionamiento de cada uno complementa a o requiere de otros, además de que sus objetivos están relacionados como estrategia de mercadotecnia.


\subsection*{Repositorio de Paquetes de software}
\subsection*{Sistema de Gestión de Paquetes de software}

\textit{paquetes de sofware}, y decimos ``usualmente'' y ``no siempre'',  i
porque se trata de estándares de facto (de uso) y no una regla técnica. 

\begin{itemize}
\item \texttt{tar.gz} Puede contener casi cualquier cosa, pero cuando hablamos
de paquetes de software en general se utilizan para distribuir \textit{archivos
fuente}, que deberán ser posteriormente \textit{compilados} como paso previo 
a ser efectivamente ejecutados. 
\item \texttt{.deb o .rpm} En general se usa
\end{itemize}


\section*{Licencia}

Este texto fue creado por Miriam Tamara Lechner y se encuentra bajo 
Licencia Creative Commons Atribución-CompartirDerivadasIgual 3.0 Unported

\end{document}
