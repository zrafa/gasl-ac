%%% LaTeX Template: Article/Thesis/etc. with colored headings and special fonts
%%%
%%% Source: http://www.howtotex.com/

\documentclass[12pt]{article}


\usepackage{apuntes-estilo}
\usepackage{fancyhdr,lastpage}
\usepackage{color,colortbl}
\usepackage{verbatim}

\def\maketitle{

% Titulo 
 \makeatletter
 {\color{bl} \centering \huge \sc \textbf{
 Procesos \\ 
\large \vspace*{-8pt} \color{black} Guía básica administración de procesos. 
 \vspace*{8pt} }\par}
 \makeatother


% Autor
 \makeatletter
 {\centering \small 
 	Departamento de Ingeniería de Computadoras \\
 	Facultad de Informática - Universidad Nacional del Comahue \\
 	\vspace{20pt} }
 \makeatother

}

% Custom headers and footers
\fancyhf{} % clear all header and footer fields
\fancypagestyle{plain}{\fancyhf{}}
  	\pagestyle{fancy}
 	\lhead{\footnotesize Administración de procesos - Departamento de Ingeniería de Computadoras}
 	\rhead{\footnotesize \thepage\ }	% ''Page 1 of 2''

\def\ti#1#2{\texttt{#1} & #2 \\ }



\begin{document}

\thispagestyle{empty}
\maketitle
\setlength{\parindent}{0pt}

\section*{Introducción}

Hemos hablado durante un tutorial anterior acerca de reconocimiento de recursos 
físicos, identificar sus características y representación dentro del 
sistema operativo.

Durante este texto nos dedicaremos a observar el consumo de esos recursos 
por parte de los procesos en ejecución. 

Monitorear un recurso significa verificar su estado, es decir si esta 
operando dentro de los parámetros normales. En algunos casos, esto se 
traduce solamente en saber si el recurso está funcionando o no, mientras 
que en otros la definición de operación normal puede ser menos clara. 
Por ejemplo en el caso de un CPU, además de si está presente o no, nos 
interesará saber cuál es su carga de procesos en ejecución. 


\section*{Procesos}

Un proceso es un programa en ejecución. Son digamos, las entidades vivas 
dentro de nuestro sistema. Cada programa que ejecutamos, por ejemplo el 
navegador web, puede disparar la ejecución de uno o mas procesos para 
llevar a cabo su tarea.  A su vez, existen procesos que no son disparados 
por los usuarios, sino que los ejecuta el sistema operativo en sí mismo 
para cumplir sus funciones. Por ejemplo, el proceso \texttt{init}, ó el 
servicio de ssh. 

Los procesos en ejecución necesitan consumir recursos computacionales para 
llevar a cabo su función: CPU, memoria principal (RAM), memoria secundaria, 
transferir datos por red, etc. Algunos procesos necesitarán más tiempo de 
CPU, como puede ser un juego, o un programa de cáluclo; mientras que 
otros, por ejemplo el web browser, necesitarán más memoria principal 
(RAM), etc.

Es la intención de este apunte observar el consumo de recursos por parte de 
los procesos en ejecución. 
 
\subsection*{Estado de los procesos}

El orden en que los procesos reciben la atención de el CPUs, es 
similar a lo que ocurre con las cajas de cobro en un supermercado. 
Imaginemos que cada caja de cobro es un CPU y las personas son los 
procesos. Al igual que en el supermercado, las colas se organizan con 
prioridades, de modo que algunos procesos pueden tener prioridad 
sobre otros a la hora de ser atendidos por el CPU. 

Los procesos no siempre se encuentran en ejecución. 
Como bien sabemos, la cantidad de CPUs disponibles es finita, por ende 
los procesos compiten por 
pequeñas porciones de tiempo de CPU. Momento en el cual, el proceso en 
ejecución tendrá la oportunidad de ejecutar algunas instrucciones de su 
código, hasta en tanto se acabe su porción de tiempo de CPU o bien ocurra 
algún otro evento que evite que el proceso se siga ejecutando (por ejemplo, 
porque está esperando recibir datos de la red). 

Si un proceso necesita abrir un archivo que se encuentra en 
memoria secundaria (disco), habrá una pequeña cantidad de tiempo en la que 
dicho proceso estará esperando por la respuesta del disco, que es muchas 
veces más lenta que la velocidad de procesamiento del CPU. Durante esta
espera, el proceso no está listo para ser ejecutado en CPU, se encuentra 
``durmiendo''. Por lo que, mientras el proceso espera, otros procesos 
pueden aprovechar y ejecutarse en el CPU. 
  
Es así que durante el tiempo de vida de un proceso, este se encontrará en 
diferentes estados. En un sistema GNU/Linux observaremos básicamente 
cinco posibles estados, observables por el adminsitrador de sistemas: 

\begin{itemize}
\item R = Corriendo (running): significa que el proceso se encuentra en 
condiciones de ejecutarse en CPU o bien que está siendo ejecutado.  
\item S = Durmiendo (sleeping): el proceso se encuentra esperando por 
agluna señal o evento, no se encuentra en condiciones de ser ejecutado en 
CPU al momento. 
\item T = Trazado o detenido (traced or stopped): proceso que ha recibido 
la orden de detenerse (por ejemplo a pedido del usuario con el comando 
kill y la señal SIGSTOP). 
\item D = Durmiendo no interrumpible (uninterruptible sleep): se encuentra
esperando por un evento y no puede ser interrumpido. 
\item Z = zombie: proceso que termino y no fue reclamado por su proceso 
padre. 
\end{itemize}

Las trancisiones entre estados siguen el siguiente esquema: 

\begin{center}
\includegraphics{process-states-s.jpg}
\end{center}
 
Estas transiciones serán en la mayoría de los casos generadas por la misma
funcionalidad del programa, es decir cuando necesita abrir un archivo
estará en estado S, cunado necesita ejecutarse estará en estado R, etc. 
Es decir, el administrador de sistemas, en general no inverviene en las
transiciones de estados. Sin embargo, hay casos en los que esto ocurre. 

\subsection*{Manipulación de procesos}

En el uso cotidiano de un sistema de escritorio, el usuario interviene 
en los estados de los procesos sin darse cuenta. Al pedirle a un programa
que se cierre (por ejemplo apretando la ``X'' en la ventanita), el usuario
le esta pidiendo al proceso que termine y deje de existir. 

El administrador de sistemas en ocasiones deberá manipular el estado de 
los procesos con diferentes fines. Para ello, enviará señales a los 
mismos a través del comando \texttt{kill}. 

\fcolorbox{black}{grey}{
\parbox[t]{1.0\linewidth}{ \vspace*{0.4cm}
{\bf Importante}: cada proceso tiene un usuario dueño (el nombre del
 usuario dueño se observa en la primer columna de la salida de 
\texttt{ps -ef}). El envío de señales a un proceso mediante el 
comando \texttt{kill} sólo pordá realizarlo el dueño o \textbf{root}
\vspace*{0.4cm} } }

El comando \texttt{kill} (matar) tiene una funcionalidad predeterminada, 
equivalente a cerrar un programa desde la interfase gráfica, esto es envía
al proceso la señal SIGTERM (15). El comando 
recibe como argumento una señal y un identificador de proceso (PID) al 
cual enviarle la señal. 

\colorbox{grey}{\parbox[t]{0.95\linewidth}{ \vspace*{0.5cm} { 
{\bf Ejemplo de uso de kill :} \\
{\tt
Sección de la salida del comando \texttt{ps -ef}\\
lechnerm  7324  7320  0 12:40 ?        00:00:00 gnome-pty-helper \\
lechnerm  7325  7320  0 12:40 pts/0    00:00:00 bash\\
lechnerm 10400  7320  0 13:11 pts/2    00:00:00 bash\\
www-data 10725  4250  0 13:12 ?        00:00:00 /usr/sbin/apache2 -k start \\
}
Si, observando la salida anterior, ejecutamos ``\texttt{kill 10725}'', 
estaremos pidiéndole al programa apache2 que termine su ejecución. 
} \vspace*{0.5cm} } } 
	
El comando \texttt{kill} permite, además de su funcionalidad predeterminada,
enviar otras señales que generan otros cambios de estado diferentes al de 
terminar. La lista de señales posibles se pueden listar observando la 
salida del comando \texttt{kill -l}. La explicación particular de cada 
una de ellas se puede encontrar en la sección siete de las paginas del 
manual online página ``signal'' (\texttt{man 7 signal}).  

Cada señal tiene un número y un nombre en mayúsculas asociado, cualquiera
de ellos puede utilizarce como argumento del comando \texttt{kill}  
En particular hablaremos de las señales mas frecuentemente utilizadas: 
SIGSTOP (19), SIGCONT(18), SIGTERM (15) y  SIGKILL (9).  

\textbf{SIGTERM (15) - Predeterminada}

La señal SIGTERM es la predeterminada al utilizar el comando kill. Esto 
es, si ejecutamos el comando kill utilizando sólo como argumento el PID 
de un proceso, le estaremos enviando SIGTERM. Esto significa que le estamos
pidiendo \textit{amablemente} al proceso que finalice su ejecución. 
Si el proceso puede responder a esta señal, ejecutará instrucciones de 
cierre (como por ejemplo cerrar archivos abiertos) y finalizará su 
ejecución correctamente liberando los recursos que estaba consumiendo. 

\colorbox{grey}{\parbox[t]{0.95\linewidth}{ \vspace*{0.5cm} { 
{\bf }
Ejecutar ``\texttt{kill 1232}'' es equivalente a ``\texttt{kill -SIGTERM 
1232}'', y a ``\texttt{kill -15 1232}'' (asumiendo que 1232 es un PID 
válido). 
} \vspace*{0.5cm} } } 

\textbf{SIGKILL (9) - Terminación abrupta}

Muchos administradores de sistemas utilizan esta señal como un estándar, 
esto además de incorrecto es riesgoso. La señal SIGKILL es equivalente en 
intención a SIGTERM, es decir que si enviamos esta señal a un proceso, 
nuestra intención es finalizar al mismo. Sin embargo, en este caso el
cierre no es amable, al enviar la señal SIGKILL el proceo pierde la 
posibilidad de finalizar correctamente y muchos de los recursos que 
utilizaba pueden quedar retenidos por un cierto período de tiempo.  


\fcolorbox{black}{grey}{
\parbox[t]{1.0\linewidth}{ \vspace*{0.4cm}
{\bf Lo importante:} La señal SIGKILL no puede ser desatendida por el 
proceso, mientras que SIGTERM si (el proceso puede no hacer caso al 
pedido de terminación). Es importante primero intentar finalizarlo 
amablemente (SIGTERM) y luego si no responde aplicar SIGKILL. 
\vspace*{0.4cm} } }


\textbf{SIGSTOP (19) - Deteniendo un proceso}

A veces, el administrador no quiere terminar un proceso, pero necesita
que éste detenga su ejecución temporalmente. La señal SIGSTOP, no puede
ser ignorada por los procesos, y realiza exactamente esta función de 
detener la ejecución del mismo. 

Una vez que enviamos esta señal a un proceso, el proceso se encontrará en 
estado detenido ``T''.


\colorbox{grey}{\parbox[t]{0.95\linewidth}{ \vspace*{0.5cm} { 
{\bf Ejemplo:} deteniendo al navegador web iceweasel. \\
{\tt
\# ps -elf|grep iceweasel |grep -v grep \\
0 S lechnerm 19308  3623 10  80   0 - 213762 -     23:49 ?    00:00:06 iceweasel \\

\# kill -SIGSTOP 19308 \\

\# ps -elf|grep iceweasel |grep -v grep \\
0 T lechnerm 19308  3623 14  80   0 - 234269 -     23:49 ?    00:00:08 iceweasel \\
}
Observese el cambio de estado (segundo valor mostrado en la salida de ps), 
de ``S'' a ``T''. 
} \vspace*{0.5cm} } } 


\textbf{SIGCONT (18) - Reactivando un proceso}

La señal SIGCONT permite al administrador indicarle al proceso que debe 
continuar su ejecución. Esta señal tiene sentido luego de haber enviado 
SIGSTOP al proceso. Siguiendo el ejemplo anterior, enviaremos la señal de
continuación, SIGCONT, al navegador web previamente detenido.  


\colorbox{grey}{\parbox[t]{0.95\linewidth}{ \vspace*{0.5cm} { 
{\bf Ejemplo:} reactivando el navegador web. \\
{\tt
\# ps -elf|grep iceweasel |grep -v grep  \\
0 T lechnerm 19308  3623 14  80   0 - 234269 -     23:49 ?    00:00:08 iceweasel \\

kill -SIGCONT 19308 \\

ps -elf|grep iceweasel |grep -v grep  \\
0 S lechnerm 19308  3623 10  80   0 - 213762 -     23:49 ?    00:00:09 iceweasel \\
}
Observese el cambio de estado (segundo valor mostrado en la salida de ps), 
de ``T'' a ``S''. 

} \vspace*{0.5cm} } }

\subsection*{Ejecución en primer y segundo plano}

En Linux proceso puede estar en primer (foreground) o segundo plano
(background). Solo un proceso estará en primer plano al mismo tiempo en
una terminal, y es con el que estemos trabajando e interactuando en ese 
momento. Un proceso que este en segundo plano no recibirá interacción  
de parte nuestra, es decir que no nos podemos comunicar con él 
a través, por ejemplo, del teclado. 

La utilidad de enviar un programa a 
segundo plano esta dada por el hecho de que existen tareas que no 
requieren de nuestro control para que se ejecuten. Por ejemplo, bajar 
algún archivo de Internet, compilar el kernel u otro programa. Estas son 
tareas que pueden ser lanzadas tranquilamente en segundo plano. También 
podemos querer realizar otra tarea en la misma terminal


Para 
lanzar un proceso en segundo plano, tendremos que poner a continuación 
del comando el símbolo ``\&''. 

En linux podemos ejecutar procesos en primer plano (foreground) o bien en segundo plano (background).

Un programa en foreground lanzado desde una terminal monopoliza dicha
terminal, por lo que en principio, no podremos ejecutar ningún otro 
programa a la vez. 

Por el contrario un programa en background una vez iniciado, deja de 
monopolizar la terminal desde el que se lanzo, y esta nos vuelve a 
mostrar el prompt.

{\bf ¿Cuándo lanzaremos un programa en background?}

Por ejemplo, en una terminal gráfica lanzamos gimp y queremos realizar 
otras operaciones desde la misma terminal, o bien vamos a lanzar un 
programa que no necesita interacción con el usuario. 


{\bf ¿Cuándo lanzaremos un programa en foreground?}

Con un proceso que necesita interacción con el usuario, y esta 
interacción se realiza a través del terminal.

{\bf ¿Como podemos lanzar otro programa desde un terminal con otro programa 
en ejecución en foreground?}

Una vez ejecutado el programa que captura la terminal, pulsamos CTRL+z 
en la misma, con lo que detenemos el programa en ejecución (equivalente
a enviar la señal SIGSTOP). Ojo lo pausamos con lo cual dejará de 
funcionar, y ya podremos lanzar otro programa p.e. ls

    Podemos hacer una prueba lanzamos gimp y comprobamos que podemos operar con el, luego pulsamos CTRL-z y vemos como dejamos de poder trabajar con gimp).

Ahora queremos volver a poner en funcionamiento a gimp y así poder volver a utilizar gimp

    Si queremos devolverlo a foreground escribiremos fg.
    Si queremos devolverlo a background escribiremos bg (esta sería la opción mas lógica)

En el caso de que tengamos mas de un programa detenido deberemos indicarle tanto a fg como a bg el ID de tarea sobre el que actuarán, este ID podemos obtenerlo con jobs que hemos visto en un apartado anterior
¿Como lanzar un programa directamente en background - &?

Siguiendo nuestro ejemplo con gimp seria gimp & . El & le indica a S.O. que ejecute el programa en segundo plano

\subsection*{Atajos en la terminal}
 


\subsection*{Otros comandos}




\section*{Referencias}

[Intro] Tutuorial Introducción a la Administración de Sistemas GNU/Linux. Materia Introducción
a la Administración de Sistemas. UNCOMA. TUASSL. 

Señales: http://tldp.org/LDP/Bash-Beginners-Guide/html/sect\_12\_01.html

http://www.ant.org.ar/cursos/curso\_intro/x1845.html

\section*{Licencia}

Este texto fue creado por Miriam Tamara Lechner y se encuentra bajo 
Licencia Creative Commons Atribución-CompartirDerivadasIgual 3.0 Unported

\end{document}
