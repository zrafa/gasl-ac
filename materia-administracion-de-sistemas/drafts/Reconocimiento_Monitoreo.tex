%%% LaTeX Template: Article/Thesis/etc. with colored headings and special fonts
%%%
%%% Source: http://www.howtotex.com/

\documentclass[12pt]{article}


\usepackage{apuntes-estilo}
\usepackage{fancyhdr,lastpage}
\usepackage{color,colortbl}
\usepackage{verbatim}

\def\maketitle{

% Titulo 
 \makeatletter
 {\color{bl} \centering \huge \sc \textbf{
 Manteniendo la hora \\ 
\large \vspace*{-8pt} \color{black} Guía básica de reconocimiento y monitoreo de recursos. 
 \vspace*{8pt} }\par}
 \makeatother


% Autor
 \makeatletter
 {\centering \small 
 	Departamento de Ingeniería de Computadoras \\
 	Facultad de Informática - Universidad Nacional del Comahue \\
 	\vspace{20pt} }
 \makeatother

}

% Custom headers and footers
\fancyhf{} % clear all header and footer fields
\fancypagestyle{plain}{\fancyhf{}}
  	\pagestyle{fancy}
 	\lhead{\footnotesize Reconocimiento y monitoreo de recursos - Departamento de Ingeniería de Computadoras}
 	\rhead{\footnotesize \thepage\ }	% ''Page 1 of 2''

\def\ti#1#2{\texttt{#1} & #2 \\ }



\begin{document}

\thispagestyle{empty}
\maketitle
\setlength{\parindent}{0pt}

\section*{Introducción}

Un administrador de sistemas es quien administra los recursos de un sistema informático. El administrador
de sistemas debe conocer cuáles son los recursos a administrar: cómo identificarlos y verificar su 
correcto funcionamiento. Durante esta guía se verán una serie de comandos y procedimientos para identificar 
recursos y verificar su estado. Dada la variedad del hardware existente hoy en día, este apunte no pretende
ser exhaustivo sino plantear un método de identificación y monitoreo a partir de ejemplos. Cunado 
el recurso a administrar no este dentro de lo visto en esta guía el administrador deberá preguntarse e
investigar cuál es la forma de identificar y observar el estado del recurso en cuestión. 

Cuando hablamos de recursos en general nos referimos a representaciones en el sistema operativo para 
recursos físicos, como por ejemplo un CPU. O bien a recursos netamente lógicos que no tienen una
contraparte física como puede ser un proceso (programa en ejecución), una prioridad de ejecución, 
un sistema de archivos, etc. 


\section*{Identificando recursos}


Comenzaremos por identificar recursos físicos. Cuando el administrador necesita realizar una tarea
sobre un recurso, por ejemplo ver si el sistema reconoce un dispositivo usb recientemente conectado, 
debe preguntarse: ¿cuál es el comando que me permite listar información acerca de este dispositivo?. 
El administrador podrá recordar un conjunto de comandos, quizá no todos, y en caso que no lo recuerde
investigará hasta encontrar la respuesta. En nuestro ejemplo del reconocimiento de un dispositivo 
usb la pregunta se resuelve casi de manera inmediata al realizar una pequeña búsqueda entre las páginas
del manual de GNU/Linux: 

\colorbox{grey}{\parbox[t]{0.95\linewidth}{ \vspace*{0.5cm} { 
Observe que efectivamente la primera y sexta línea de resultados tienen grandes chances 
de darnos algo de la información que estamos buscando: \\ 
{\tt 
\$ man -k usb \\
lsusb (8)            - list USB devices\\
sane-find-scanner (1) - find SCSI and USB scanners and their device files\\
sisusb (4)           - SiS USB video driver\\
unetbootin (1)       - program to install Linux/BSD distributions to a partit...\\
update-usbids (8)    - download new version of the USB ID list\\
usb-devices (1)      - print USB device details\\
usb\_modeswitch (1)   - switch mode of ``multi-state'' USB devices\\
usb\_printerid (1)    - prints the ID of the printer on a USB port\\
usbip (8)            - manage USB/IP devices\\
usbipd (8)           - USB/IP server daemon\\
usbmuxd (1)          - iPhone/iPod Touch USB multiplex server daemon\\ \\
}
NOTA A PEGAR EN EL ESPEJO: ¡``{\tt man -k}'' es tu amigo!
} \vspace*{0.5cm} } } 

Obviamente si el administrador no logra resultados investigando entre las 
páginas del manual de la máquina en cuestión, irá a Internet y haciendo la consulta
correcta ``list usb devices linux'' ó ``listar usb en linux'' (en 
muchos casos la búsqueda en idioma inglés devuelva más resultados que en español),  
muy probablemente obtendrá lo que busca entre los primeros resultados: 

\begin{center}
 \includegraphics{lsusb.jpg}
\end{center}

Si no resuelve la pregunta de este modo, recurra a expertos en el tema, pero recuerde
intentar la identificación de recursos por sus propios medios al menos siguiendo los dos 
métodos anteriores, el administrador debe ser curioso e investigar, de otro modo puede 
obtener respuestas como esta: http://lmgtfy.com/?q=list+usb+devices+linux

\subsection*{Identificando el hardware}
En esta sección listaremos una serie de comandos clásicos para identificar recursos de
hardware dentro del sistema operativo. 

\subsection*{Identificando recursos netamente lógicos}

\section*{Monitoreo de recursos}


\colorbox{grey}{\parbox[t]{0.95\linewidth}{ \vspace*{0.5cm} { 
{\tt 
11) none - I want to specify the time zone using the Posix TZ format.\\
\#? 1\\
}
} \vspace*{0.5cm} } } 




\section*{Licencia}

Este material es una obra derivada de los siguientes textos:

``The Clock Mini-HOWTO'' del sitio TLDP: http://tldp.org/HOWTO/Clock.html\#toc1

``The Linux System Administrator's Guide'' del sitio TLDP: http://www.tldp.org/LDP/sag/html/

\end{document}
